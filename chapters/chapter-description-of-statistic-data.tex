\chapter{统计数据的描述}

使用图形描述统计结果是应用统计的基本技能之一。人类在接受信息时,大脑皮层总是优先提取可视化的图形,其次才是具体的文字内容。一些漂亮的图形会使得读者或观众迅速得知我们想要表达的信息。因此不论是写论文,还是汇报展示,做到图文并茂非常重要。

由于 Python 具有良好的可扩展性,它在画图方面具有大量的绘图包,包括 模仿 Matlab 绘图风格的 matplotlib, 画地理图形的  geoplotlib,能在浏览器中生成优美图形的 Pyechart等。其中,matplotlib 是统计分析中应用最普遍的绘图包之一。本章首先将展示如何用 Python 整理统计数据,然后使用 matplotlib 来画出各种各样的统计图形。


\section{统计数据的整理}
\subsection{数据审核}

\subsection{数据排序}


\subsection{数据分组}




\section{统计数据的图形展示}

使用 matplotlib 包画图时,我们一般用里面的 plot 函数,加载宏包时命名为 plt。


\begin{lstlisting}[language=Python]
# 导入 matplotlib 中的 plot, 并命名为常用名 plt
import matplotlib.plot as plt 
\end{lstlisting}

例如,下面的代码画出正弦函数 $y=sin(x)$ 的图形。


\begin{lstlisting}[language=Python]
# 导入宏包
import matplotlib.pyplot as plt
import numpy as np

# 生成数据
x = np.arange(0, 10, 0.1)
y = np.sin(x)

# 生成图形
plt.plot(x, y)

# 显示图形
plt.show()

\end{lstlisting}




\subsection{线图}

\subsection{散点图}

\subsection{直方图}

\subsection{饼图}

\subsection{箱型图}

\subsection{雷达图}

\subsection{三维图形}

